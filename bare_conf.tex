
%% bare_conf_compsoc.tex
%% V1.4b
%% 2015/08/26
%% by Michael Shell
%% See:
%% http://www.michaelshell.org/
%% for current contact information.
%%
%% This is a skeleton file demonstrating the use of IEEEtran.cls
%% (requires IEEEtran.cls version 1.8b or later) with an IEEE Computer
%% Society conference paper.
%%
%% Support sites:
%% http://www.michaelshell.org/tex/ieeetran/
%% http://www.ctan.org/pkg/ieeetran
%% and
%% http://www.ieee.org/

%%*************************************************************************
%% Legal Notice:
%% This code is offered as-is without any warranty either expressed or
%% implied; without even the implied warranty of MERCHANTABILITY or
%% FITNESS FOR A PARTICULAR PURPOSE! 
%% User assumes all risk.
%% In no event shall the IEEE or any contributor to this code be liable for
%% any damages or losses, including, but not limited to, incidental,
%% consequential, or any other damages, resulting from the use or misuse
%% of any information contained here.
%%
%% All comments are the opinions of their respective authors and are not
%% necessarily endorsed by the IEEE.
%%
%% This work is distributed under the LaTeX Project Public License (LPPL)
%% ( http://www.latex-project.org/ ) version 1.3, and may be freely used,
%% distributed and modified. A copy of the LPPL, version 1.3, is included
%% in the base LaTeX documentation of all distributions of LaTeX released
%% 2003/12/01 or later.
%% Retain all contribution notices and credits.
%% ** Modified files should be clearly indicated as such, including  **
%% ** renaming them and changing author support contact information. **
%%*************************************************************************


% *** Authors should verify (and, if needed, correct) their LaTeX system  ***
% *** with the testflow diagnostic prior to trusting their LaTeX platform ***
% *** with production work. The IEEE's font choices and paper sizes can   ***
% *** trigger bugs that do not appear when using other class files.       ***                          ***
% The testflow support page is at:
% http://www.michaelshell.org/tex/testflow/

%\documentclass[journal,onecolumn]{IEEEtran}
\documentclass[conference,compsoc]{IEEEtran}

% Some/most Computer Society conferences require the compsoc mode option,
% but others may want the standard conference format.
%
% If IEEEtran.cls has not been installed into the LaTeX system files,
% manually specify the path to it like:
% \documentclass[conference,compsoc]{../sty/IEEEtran}





% Some very useful LaTeX packages include:
% (uncomment the ones you want to load)


% *** MISC UTILITY PACKAGES ***
%
%\usepackage{ifpdf}
% Heiko Oberdiek's ifpdf.sty is very useful if you need conditional
% compilation based on whether the output is pdf or dvi.
% usage:
% \ifpdf
%   % pdf code
% \else
%   % dvi code
% \fi
% The latest version of ifpdf.sty can be obtained from:
% http://www.ctan.org/pkg/ifpdf
% Also, note that IEEEtran.cls V1.7 and later provides a builtin
% \ifCLASSINFOpdf conditional that works the same way.
% When switching from latex to pdflatex and vice-versa, the compiler may
% have to be run twice to clear warning/error messages.






% *** CITATION PACKAGES ***
%
\ifCLASSOPTIONcompsoc
  % IEEE Computer Society needs nocompress option
  % requires cite.sty v4.0 or later (November 2003)
  \usepackage[nocompress]{cite}
\else
  % normal IEEE
  \usepackage{cite}
\fi
% cite.sty was written by Donald Arseneau
% V1.6 and later of IEEEtran pre-defines the format of the cite.sty package
% \cite{} output to follow that of the IEEE. Loading the cite package will
% result in citation numbers being automatically sorted and properly
% "compressed/ranged". e.g., [1], [9], [2], [7], [5], [6] without using
% cite.sty will become [1], [2], [5]--[7], [9] using cite.sty. cite.sty's
% \cite will automatically add leading space, if needed. Use cite.sty's
% noadjust option (cite.sty V3.8 and later) if you want to turn this off
% such as if a citation ever needs to be enclosed in parenthesis.
% cite.sty is already installed on most LaTeX systems. Be sure and use
% version 5.0 (2009-03-20) and later if using hyperref.sty.
% The latest version can be obtained at:
% http://www.ctan.org/pkg/cite
% The documentation is contained in the cite.sty file itself.
%
% Note that some packages require special options to format as the Computer
% Society requires. In particular, Computer Society  papers do not use
% compressed citation ranges as is done in typical IEEE papers
% (e.g., [1]-[4]). Instead, they list every citation separately in order
% (e.g., [1], [2], [3], [4]). To get the latter we need to load the cite
% package with the nocompress option which is supported by cite.sty v4.0
% and later.





% *** GRAPHICS RELATED PACKAGES ***
%
\ifCLASSINFOpdf
  % \usepackage[pdftex]{graphicx}
  % declare the path(s) where your graphic files are
  % \graphicspath{{../pdf/}{../jpeg/}}
  % and their extensions so you won't have to specify these with
  % every instance of \includegraphics
  % \DeclareGraphicsExtensions{.pdf,.jpeg,.png}
\else
  % or other class option (dvipsone, dvipdf, if not using dvips). graphicx
  % will default to the driver specified in the system graphics.cfg if no
  % driver is specified.
  % \usepackage[dvips]{graphicx}
  % declare the path(s) where your graphic files are
  % \graphicspath{{../eps/}}
  % and their extensions so you won't have to specify these with
  % every instance of \includegraphics
  % \DeclareGraphicsExtensions{.eps}
\fi
% graphicx was written by David Carlisle and Sebastian Rahtz. It is
% required if you want graphics, photos, etc. graphicx.sty is already
% installed on most LaTeX systems. The latest version and documentation
% can be obtained at: 
% http://www.ctan.org/pkg/graphicx
% Another good source of documentation is "Using Imported Graphics in
% LaTeX2e" by Keith Reckdahl which can be found at:
% http://www.ctan.org/pkg/epslatex
%
% latex, and pdflatex in dvi mode, support graphics in encapsulated
% postscript (.eps) format. pdflatex in pdf mode supports graphics
% in .pdf, .jpeg, .png and .mps (metapost) formats. Users should ensure
% that all non-photo figures use a vector format (.eps, .pdf, .mps) and
% not a bitmapped formats (.jpeg, .png). The IEEE frowns on bitmapped formats
% which can result in "jaggedy"/blurry rendering of lines and letters as
% well as large increases in file sizes.
%
% You can find documentation about the pdfTeX application at:
% http://www.tug.org/applications/pdftex





% *** MATH PACKAGES ***
%
%\usepackage{amsmath}
% A popular package from the American Mathematical Society that provides
% many useful and powerful commands for dealing with mathematics.
%
% Note that the amsmath package sets \interdisplaylinepenalty to 10000
% thus preventing page breaks from occurring within multiline equations. Use:
%\interdisplaylinepenalty=2500
% after loading amsmath to restore such page breaks as IEEEtran.cls normally
% does. amsmath.sty is already installed on most LaTeX systems. The latest
% version and documentation can be obtained at:
% http://www.ctan.org/pkg/amsmath





% *** SPECIALIZED LIST PACKAGES ***
%
\usepackage{amsmath}
\usepackage{algorithm}
\usepackage{algorithmicx}
\usepackage[noend]{algpseudocode}
% algorithmic.sty was written by Peter Williams and Rogerio Brito.
% This package provides an algorithmic environment fo describing algorithms.
% You can use the algorithmic environment in-text or within a figure
% environment to provide for a floating algorithm. Do NOT use the algorithm
% floating environment provided by algorithm.sty (by the same authors) or
% algorithm2e.sty (by Christophe Fiorio) as the IEEE does not use dedicated
% algorithm float types and packages that provide these will not provide
% correct IEEE style captions. The latest version and documentation of
% algorithmic.sty can be obtained at:
% http://www.ctan.org/pkg/algorithms
% Also of interest may be the (relatively newer and more customizable)
% algorithmicx.sty package by Szasz Janos:
% http://www.ctan.org/pkg/algorithmicx




% *** ALIGNMENT PACKAGES ***
%
%\usepackage{array}
% Frank Mittelbach's and David Carlisle's array.sty patches and improves
% the standard LaTeX2e array and tabular environments to provide better
% appearance and additional user controls. As the default LaTeX2e table
% generation code is lacking to the point of almost being broken with
% respect to the quality of the end results, all users are strongly
% advised to use an enhanced (at the very least that provided by array.sty)
% set of table tools. array.sty is already installed on most systems. The
% latest version and documentation can be obtained at:
% http://www.ctan.org/pkg/array


% IEEEtran contains the IEEEeqnarray family of commands that can be used to
% generate multiline equations as well as matrices, tables, etc., of high
% quality.




% *** SUBFIGURE PACKAGES ***
%\ifCLASSOPTIONcompsoc
%  \usepackage[caption=false,font=footnotesize,labelfont=sf,textfont=sf]{subfig}
%\else
%  \usepackage[caption=false,font=footnotesize]{subfig}
%\fi
% subfig.sty, written by Steven Douglas Cochran, is the modern replacement
% for subfigure.sty, the latter of which is no longer maintained and is
% incompatible with some LaTeX packages including fixltx2e. However,
% subfig.sty requires and automatically loads Axel Sommerfeldt's caption.sty
% which will override IEEEtran.cls' handling of captions and this will result
% in non-IEEE style figure/table captions. To prevent this problem, be sure
% and invoke subfig.sty's "caption=false" package option (available since
% subfig.sty version 1.3, 2005/06/28) as this is will preserve IEEEtran.cls
% handling of captions.
% Note that the Computer Society format requires a sans serif font rather
% than the serif font used in traditional IEEE formatting and thus the need
% to invoke different subfig.sty package options depending on whether
% compsoc mode has been enabled.
%
% The latest version and documentation of subfig.sty can be obtained at:
% http://www.ctan.org/pkg/subfig




% *** FLOAT PACKAGES ***
%
%\usepackage{fixltx2e}
% fixltx2e, the successor to the earlier fix2col.sty, was written by
% Frank Mittelbach and David Carlisle. This package corrects a few problems
% in the LaTeX2e kernel, the most notable of which is that in current
% LaTeX2e releases, the ordering of single and double column floats is not
% guaranteed to be preserved. Thus, an unpatched LaTeX2e can allow a
% single column figure to be placed prior to an earlier double column
% figure.
% Be aware that LaTeX2e kernels dated 2015 and later have fixltx2e.sty's
% corrections already built into the system in which case a warning will
% be issued if an attempt is made to load fixltx2e.sty as it is no longer
% needed.
% The latest version and documentation can be found at:
% http://www.ctan.org/pkg/fixltx2e


%\usepackage{stfloats}
% stfloats.sty was written by Sigitas Tolusis. This package gives LaTeX2e
% the ability to do double column floats at the bottom of the page as well
% as the top. (e.g., "\begin{figure*}[!b]" is not normally possible in
% LaTeX2e). It also provides a command:
%\fnbelowfloat
% to enable the placement of footnotes below bottom floats (the standard
% LaTeX2e kernel puts them above bottom floats). This is an invasive package
% which rewrites many portions of the LaTeX2e float routines. It may not work
% with other packages that modify the LaTeX2e float routines. The latest
% version and documentation can be obtained at:
% http://www.ctan.org/pkg/stfloats
% Do not use the stfloats baselinefloat ability as the IEEE does not allow
% \baselineskip to stretch. Authors submitting work to the IEEE should note
% that the IEEE rarely uses double column equations and that authors should try
% to avoid such use. Do not be tempted to use the cuted.sty or midfloat.sty
% packages (also by Sigitas Tolusis) as the IEEE does not format its papers in
% such ways.
% Do not attempt to use stfloats with fixltx2e as they are incompatible.
% Instead, use Morten Hogholm'a dblfloatfix which combines the features
% of both fixltx2e and stfloats:
%
% \usepackage{dblfloatfix}
% The latest version can be found at:
% http://www.ctan.org/pkg/dblfloatfix




% *** PDF, URL AND HYPERLINK PACKAGES ***
%
%\usepackage{url}
% url.sty was written by Donald Arseneau. It provides better support for
% handling and breaking URLs. url.sty is already installed on most LaTeX
% systems. The latest version and documentation can be obtained at:
% http://www.ctan.org/pkg/url
% Basically, \url{my_url_here}.




% *** Do not adjust lengths that control margins, column widths, etc. ***
% *** Do not use packages that alter fonts (such as pslatex).         ***
% There should be no need to do such things with IEEEtran.cls V1.6 and later.
% (Unless specifically asked to do so by the journal or conference you plan
% to submit to, of course. )

\usepackage{caption}
\captionsetup[table]{position=bottom}


% correct bad hyphenation here
\hyphenation{op-tical net-works semi-conduc-tor}


\begin{document}
%
% paper title
% Titles are generally capitalized except for words such as a, an, and, as,
% at, but, by, for, in, nor, of, on, or, the, to and up, which are usually
% not capitalized unless they are the first or last word of the title.
% Linebreaks \\ can be used within to get better formatting as desired.
% Do not put math or special symbols in the title.
\title{A Survey on Near Data Processing Accelerators for Graph Analytics}


% author names and affiliations
% use a multiple column layout for up to three different
% affiliations
\author{\IEEEauthorblockN{Makesh Tarun Chandran}\\
\IEEEauthorblockA{Department of Electrical Engineering and Computer Science\\
Pennsylvania State University\\
Email: mzc88@psu.edu}}
% \and
% \IEEEauthorblockN{Homer Simpson}
% \IEEEauthorblockA{Twentieth Century Fox\\
% Springfield, USA\\
% Email: homer@thesimpsons.com}
% \and
% \IEEEauthorblockN{James Kirk\\ and Montgomery Scott}
% \IEEEauthorblockA{Starfleet Academy\\
% San Francisco, California 96678-2391\\
% Telephone: (800) 555--1212\\
% Fax: (888) 555--1212}}

% conference papers do not typically use \thanks and this command
% is locked out in conference mode. If really needed, such as for
% the acknowledgment of grants, issue a \IEEEoverridecommandlockouts
% after \documentclass

% for over three affiliations, or if they all won't fit within the width
% of the page (and note that there is less available width in this regard for
% compsoc conferences compared to traditional conferences), use this
% alternative format:
% 
%\author{\IEEEauthorblockN{Michael Shell\IEEEauthorrefmark{1},
%Homer Simpson\IEEEauthorrefmark{2},
%James Kirk\IEEEauthorrefmark{3}, 
%Montgomery Scott\IEEEauthorrefmark{3} and
%Eldon Tyrell\IEEEauthorrefmark{4}}
%\IEEEauthorblockA{\IEEEauthorrefmark{1}School of Electrical and Computer Engineering\\
%Georgia Institute of Technology,
%Atlanta, Georgia 30332--0250\\ Email: see http://www.michaelshell.org/contact.html}
%\IEEEauthorblockA{\IEEEauthorrefmark{2}Twentieth Century Fox, Springfield, USA\\
%Email: homer@thesimpsons.com}
%\IEEEauthorblockA{\IEEEauthorrefmark{3}Starfleet Academy, San Francisco, California 96678-2391\\
%Telephone: (800) 555--1212, Fax: (888) 555--1212}
%\IEEEauthorblockA{\IEEEauthorrefmark{4}Tyrell Inc., 123 Replicant Street, Los Angeles, California 90210--4321}}




% use for special paper notices
%\IEEEspecialpapernotice{(Invited Paper)}




% make the title area
\maketitle

% As a general rule, do not put math, special symbols or citations
% in the abstract

%\begin{abstract}

%\end{abstract}

% no keywords



  
% For peer review papers, you can put extra information on the cover
% page as needed:
% \ifCLASSOPTIONpeerreview
% \begin{center} \bfseries EDICS Category: 3-BBND \end{center}
% \fi
%
% For peerreview papers, this IEEEtran command inserts a page break and
% creates the second title. It will be ignored for other modes.
\IEEEpeerreviewmaketitle

\begin{abstract}
 
  Graph algorithms are ubiquitous, from social science to machine learning analysis. They are also notorious for performing poorly in state-of-the-art computing systems, requiring more compute capability than they need. This is due to the memory bandwidth limitations of the current computing systems. In this regard, the Processing-In-Memory technology shows potential in tackling the bottleneck. A few graph analytics accelerators have been proposed based on the in-memory computing technology in recent years. This paper attempts to summarize the basic graph programming models, various in-memory technologies, and finally discuss the recently developed graph accelerator's endeavor at alleviating various performance bottlenecks.
  
\end{abstract}


\section{Introduction}
% no \IEEEPARstart
Graph algorithms are an interesting class of application for which a single mac mini desktop equipped with SSD outperforms a medium sized cluster! \cite{GraphChi} The primary reason for such a counter-intuitive behavior can be attributed to the irregular data-access pattern inherent in those algorithms. They also have a low computation to communication ratio which is exacerbated in the modern memory-bandwidth limited systems. However, graph algorithms are highly iterative and there is abundance of parallelism that can be exploited. These idiosyncrasies of graph algorithm render commercial HPC systems inefficient or unsuitable for accelerating such applications. In other words, there's a lot of scope for deriving performance improvements from a processor customised for graph processing. Extensive research on the topic in the past decade have revealed new insights on the algorithmic, programming model and hardware aspects of such systems. From the hardware perspective, emergence of in-memory computing technology has enabled us to overcome the limitation of the conventional Von-Neumann architecture. Logic units can now be integrated into the memory cells, evading the dreaded off-chip communication. Such technologies have given birth to Processing-In-Memory (PIM) based accelerators for graph analytics. 

The rest of the paper is organised as follows. Two basic graph processing techniques are discussed in Section 2. Section 3 briefly introduces PIM techniques on different memory technologies. Section 4 discusses some of the state-of-the-art graph analytics accelerators building on the technologies described in Section 2 \& 3.



\section{Graph Processing}

Bring out the importance of Graph processing from \cite{BrainGraph,RoadGraph, SocialGraph} and discuss state-of-the-art systems. Take the example of SSSP application and discuss how the Implementation changes with various models.


  \subsection{Programming model}
    Discuss the Vertex-based \cite{Pregel} and Gather-Apply-Scatter\cite{PowerGraph} based programming models here. Highlight the differences and bring out the suitability of each of them for PIM operations. Also show the pseudo code for SSSP application.

  \subsection{Execution Model}
    Two different paradigms: Synchronous vs Asynchronous\cite{Tesseract}, Ordered vs Unordered\cite{OvsUO}


\section{In/Near-Memory Processing}
- Reduce the cost of data movement
- PIM Analog chracteristics


  \subsection{Near-Data Processing}
  Discuss \cite{Biscuit} like architectures where the computations are moved closer to the memory elements but not inside them.
  End the section by indicating that more bandwidth can be exploited when the processing is integrated with memory.
  - User managed memory layouting graphicianado, biscuit, etc
  
  \subsection{In-memory Processing}
    Non-vonneuman architectures
    \subsubsection{SRAM-PIM}
    
    \subsubsection{DRAM-PIM}
      Present a basic idea of the \cite{AMC} paper which proposes a architecture for processing-in-memory architecture built on commercially demonstrated Hybrid Memory Cube. Also highlight the loopholes/assumptions with such design. Also list other architectures, like GraphH\cite{GraphH}, which builds on the technology.

    \subsubsection{NVM-PIM}
      Briefly list the problems with DRAM in terms of energy and performance, and bring out NVMs as an alternative to DRAM.
      \cite{MagicNOR} introduces a efficient NOR operation which can be performed on the bitlines of ReRAM based memory. Show example architecture of NVM PIN and explain it working with relevant figures.


\section{Accelerators}
An Ideal graph processing system is where the performance increases proportional to the size of the graphs that can be stored in the system. Unfortunately, in conventional systems, memory bandwidth remains almost constant irrespective of the memory capacity due to pin count limitation per chip. Traditional caching mechanisms fail to sustain the memory throughput requirement due to the application's poor locality. Following is the summary of accelerators with unconventional memory design, developed to mitigate the performance bottleneck in graph processing systems. \par

\begin{table*}[t]
\small
\centering
\caption{Comparison of various features of Graph Analytics Accelerators}
\begin{tabular}{|p{0.13\linewidth}|p{0.14\linewidth}|p{0.14\linewidth}|p{0.14\linewidth}|p{0.14\linewidth}|p{0.14\linewidth}|}
\hline
& \textbf{Graphicianado}                 & \textbf{GraphH}      & \textbf{Tesseract}  & \textbf{GraphR}& \textbf{GraphP} \\ \hline

\textbf{Programing Model} & Gather-Apply-Scatter                   & Gather-Apply-Broadcast            & Pregel-like                & Pregel-like            & Pregel like + Two phase Update              \\ \hline

\textbf{Execution Model}  & Pipelined               & Message passing and shared memory & Message passing and iterative & Sparse matrix computation & Message passing and iterative \\ \hline

\textbf{Computing Mode}   & Sync             &    Sync                               & Sync \& Async                         &  Sync                         & Sync \& Async        \\ \hline

\textbf{Memory Organisation}      & Explicitly managed in Scratchpad & Memory-disk hybrid              & DRAM-PIM (HMC)                     & ReRAM-PIM                 & DRAM-PIM (HMC)                     \\ \hline

\textbf{Core} 		& Custom pipeline processor & General purpose processor&In-order core with prefetcher  & ReRam Crossbar \& ALUs  &In-order core with prefetcher \\ \hline

\textbf{Generality}       & Any Vertex program                   &    Any Vertex or Edge based program             & Any Vertex program                & Vertex program in SpMV    & Any Vertex program    \\ \hline


\end{tabular}
\end{table*}

 \subsection{Graphicionado}
 Graphicionado \cite{Graphicionado} is a custom hardware accelerator based on the Gather-Apply-Scatter model. The execution in each of the phases are broken down into pipeline stages and custom hardware units are deployed to perform the operations. There are about 10 stages in the Processing phase and 6 stages in the Update phase. A vertex can only be in one of the stage making the system inherently atomic. The memory subsystem is also redesigned to suit graph processing. Scratchpad memory is used to buffer the temporary updates to the vertices, and an edge Id table which stores the first edge of each vertex, eliminating most of the random accesses to the memory. Other optimizations such as streamed data access, and prefetching are also supported. 
 
 To scale the processing unit, the pipeline stages are split into source-oriented and destination-oriented stages and the multiple instance of each them are inter-connected through a crossbar. The parallelization of source and destination streams eliminates memory access conflicts as each of the streams access regions exclusively assigned to it. To achieve this, graph slicing techniques are used to create partitions of graphs which have minimum interaction with each other. This trivializes the scaling of the scratchpad memory system to simply creating m instances of scratchpad units for m source-and-destination streams. The accelerator achieves upto 6x speedup while consuming less than 2\% of energy of compared to the state-of-the-art software graphs analytics framework running on a 16-core Haswell Xeon server. However, for large graphs it is not feasible to hold all the relevant metadata in scratchpad and data movement across the hierarchy degrades the system performance for such large inputs.
 
 \subsection{GraphH} GraphH \cite{GraphH} is a high performance distributed big graph analytics framework optimized for a small number of clusters. It is built on a spark-based graph processing engine, a distributed file-system and MPI based graph processing engine. The underlying principle is to implement a memory-disk hybrid approach which reduces the data movement overhead by maximising the reuse of in-memory data. In this regard, three techniques are employed:
 \begin{enumerate}
  \item The graph is evenly split into tiles and distributed to servers for performing computation while the whole graph is replicated in all the systems
  \item Gather-Apply-Broadcast(GAB) is used to represent the distribution of tiles across the server. 
  \item Edge Cache system -lines are evicted only when cache is full - is employed to utilise the idle memory to reduce disk IO overhead.
 \end{enumerate}

 The difference between GAB and GAS is that GAB maintains replicas of data in every server, which have to be updated by a 'Broadcast' after every 'Apply' step. Various graph partitioning mechanisms such as Hash-based Edge-cut, Intelligent Vertex-cut and Stream partitioning are also analysed for communication and computation overheads on small clusters. Performance evaluation shows that GraphH can be 7.8X faster than the in-memory systems such as Pregel+ and Powergraph while processing generic graphs, and 100X faster than the out-of-core systems such as GraphD and Chaos.

 
 \subsection{Tesseract} The Tesseract architecture \cite{Tesseract} exploits the internal bandwidth of the HMC-RAM, which is an order of magnitude higher than the off-chip bandwidth, by integrating a logic layer within the memory die. The HMC in Tesseract is organised as vaults which are vertical slices of memory in a cube. Each vault is composed of a 16-bank DRAM partition and a dedicated memory controller. The logic layer of a vault is equipped with a single issue in-order core. Each cube can host upto 32 vaults and 8 high-speed serial links as off-chip interface. The Tesseract cores (cubes) are integrated with the host as a memory-mapped co-processor device.
 
 The host distributes jobs between the cores based on the location of data relevant to the job. During execution data access to the remote cubes are facilitated through a low cost message passing mechanism. The communication mechanism is inherently atomic and its latency can be hidden through asynchronous communication. The remote access latency can also be overcome by employing a data prefetcher which derives heuristics from the program and execution trace. The programming interface consists of a set of APIs capturing basic functions in the underlying hardware, on which any commercial graph analytics framework can be built. The proposed system was evaluated on five state-of-the-art graph processing applications and achieve a speedup of over 14x compared to the standard DDR3 based systems. Tesseract also achieves memory-capacity proportional performance, which is the key to handling increasing amounts of data in a cost-effective manner. However, when scaling the system with multiple Tesseract cores, the off-chip communication latency incurred while accessing data in a remote core becomes the performance bottleneck.
 
 \subsection{GraphR} GraphR \cite{GraphR} leverages on the fact that graph computations are inherently tolerant to imprecision and resilient to errors, to perform approximate computations on the graph, using ReRAM crossbar discussed in section \ref{gmat}. The GraphR architecture contains two key components : memory ReRAM and graph engine. The graph is stored in memresistor-based RAM (called ReRAM) in compressed sparse representation of its adjacency matrix. The graph engine is equipped with ReRAM crossbars, Analog to Digital converters, a simple ALU and IO registers to cache IO values. The schemes for mapping two prominent operations - parallel-MAC and parallel-ADD  are described in the paper. To facilitate the data movement between GE and memory, streaming-apply mechanism is employed. Graphs are stored in coloumn major order, and vertices in a coloumn are streamed onto GEs to be processed together. GraphR can be plugged in as an accelerator to a host processor which assigns a subgraph of a large graph for processing. The experiment results show that GraphR achieves 16X speedup and a 33X energy saving on geometric mean compared to a CPU baseline system. Compared to GPU, GraphR achieves 1.69X to 2.19X speedup and consumes 4.7X to 8.9X less energy. GraphR gains a speedup of upto 4X, and is 10X more energy efficiency compared to Tesseract \cite{Tesseract}. However, the whole idea rests on the assumption that the proposed memresistive device with its energy characteristics is practically feasible, which has not been demonstrated yet. Also, the SpMV based computation model limits the number of graph applications that can be efficiently supported by the proposed architecture.
 
 \subsection{GraphP} The motivation behind GraphP \cite{GraphP} is to reduce the excessive cross-cube communication overhead through SerDes links whose bandwidth is much less than the bandwidth available within a cube. The design considers data organisation as the first-order design criteria in mitigating the bottleneck. The techniques developed for the same are -
 \begin{enumerate}
  \item 'Source-Cut' partitioning - Split the graph such that all the incoming edges of a vertex are contained in the same cube.
  \item 'Two-phase vertex' programming model - \textit{GenUpdate} to generate the vertex value update on all incoming edges \& \textit{ApplyUpdate} to apply the update to each vertex. The two phases can be overlapped, with each other and the execution, to further hide the latency.
 \end{enumerate}
  With these optimisations, the proposed architecture is able to to achieve a speedup of up to 1.7x and energy efficiency of 89\% compared to Tesseract. This is at the expense of programmability of the hardware, due to the introduction of Two-phase vertex programming. Also Source-Cut partitioning may not be feasible for power law like vertex degree distribution where some vertices have large number of edges which may not fit inside a cube.
 

\section{Completed and Future Work}

The architectures listed above were proposed in the last three years, with the advent of new memory technologies which made in-memory processing feasible. An in-depth study of the hardware and software stack of these PIM-based accelerators and their design choices, may expose untapped design points to improve performance. The plan was to begin by understanding various design points listed in Section \ref{DesignSpace}, evaluate the accelerators for strengths and limitations, and identify potential improvements in their design, leading to a survey paper or an original research paper.

I have read papers GraphR\cite{GraphR}, GraphP\cite{GraphP} and other papers based on NVM-PIM \cite{k-means}. Apart from that I also ramped up with Pregel and ZSim Simulation framework. The next step would be to prototype one of the architectures (possibly GraphR or GraphH) and identify bottlenecks to improve the perforamance. A survey paper on this topic makes little sense as the in-memory processing of graphs is a relatively narrow and emerging field. Moreover, there are recent resources \cite{BGA-book, SG-MT} that contain an exhaustive analysis of the topic.


% An example of a floating figure using the graphicx package.
% Note that \label must occur AFTER (or within) \caption.
% For figures, \caption should occur after the \includegraphics.
% Note that IEEEtran v1.7 and later has special internal code that
% is designed to preserve the operation of \label within \caption
% even when the captionsoff option is in effect. However, because
% of issues like this, it may be the safest practice to put all your
% \label just after \caption rather than within \caption{}.
%
% Reminder: the "draftcls" or "draftclsnofoot", not "draft", class
% option should be used if it is desired that the figures are to be
% displayed while in draft mode.
%
%\begin{figure}[!t]
%\centering
%\includegraphics[width=2.5in]{myfigure}
% where an .eps filename suffix will be assumed under latex, 
% and a .pdf suffix will be assumed for pdflatex; or what has been declared
% via \DeclareGraphicsExtensions.
%\caption{Simulation results for the network.}
%\label{fig_sim}
%\end{figure}

% Note that the IEEE typically puts floats only at the top, even when this
% results in a large percentage of a column being occupied by floats.


% An example of a double column floating figure using two subfigures.
% (The subfig.sty package must be loaded for this to work.)
% The subfigure \label commands are set within each subfloat command,
% and the \label for the overall figure must come after \caption.
% \hfil is used as a separator to get equal spacing.
% Watch out that the combined width of all the subfigures on a 
% line do not exceed the text width or a line break will occur.
%
%\begin{figure*}[!t]
%\centering
%\subfloat[Case I]{\includegraphics[width=2.5in]{box}%
%\label{fig_first_case}}
%\hfil
%\subfloat[Case II]{\includegraphics[width=2.5in]{box}%
%\label{fig_second_case}}
%\caption{Simulation results for the network.}
%\label{fig_sim}
%\end{figure*}
%
% Note that often IEEE papers with subfigures do not employ subfigure
% captions (using the optional argument to \subfloat[]), but instead will
% reference/describe all of them (a), (b), etc., within the main caption.
% Be aware that for subfig.sty to generate the (a), (b), etc., subfigure
% labels, the optional argument to \subfloat must be present. If a
% subcaption is not desired, just leave its contents blank,
% e.g., \subfloat[].


% An example of a floating table. Note that, for IEEE style tables, the
% \caption command should come BEFORE the table and, given that table
% captions serve much like titles, are usually capitalized except for words
% such as a, an, and, as, at, but, by, for, in, nor, of, on, or, the, to
% and up, which are usually not capitalized unless they are the first or
% last word of the caption. Table text will default to \footnotesize as
% the IEEE normally uses this smaller font for tables.
% The \label must come after \caption as always.
%
%\begin{table}[!t]
%% increase table row spacing, adjust to taste
%\renewcommand{\arraystretch}{1.3}
% if using array.sty, it might be a good idea to tweak the value of
% \extrarowheight as needed to properly center the text within the cells
%\caption{An Example of a Table}
%\label{table_example}
%\centering
%% Some packages, such as MDW tools, offer better commands for making tables
%% than the plain LaTeX2e tabular which is used here.
%\begin{tabular}{|c||c|}
%\hline
%One & Two\\
%\hline
%Three & Four\\
%\hline
%\end{tabular}
%\end{table}


% Note that the IEEE does not put floats in the very first column
% - or typically anywhere on the first page for that matter. Also,
% in-text middle ("here") positioning is typically not used, but it
% is allowed and encouraged for Computer Society conferences (but
% not Computer Society journals). Most IEEE journals/conferences use
% top floats exclusively. 
% Note that, LaTeX2e, unlike IEEE journals/conferences, places
% footnotes above bottom floats. This can be corrected via the
% \fnbelowfloat command of the stfloats package.



% conference papers do not normally have an appendix



% use section* for acknowledgment
% \ifCLASSOPTIONcompsoc
%   % The Computer Society usually uses the plural form
%   \section*{Acknowledgments}
% \else
%   % regular IEEE prefers the singular form
%   \section*{Acknowledgment}
% \fi
% 
% 
% The authors would like to thank...





% trigger a \newpage just before the given reference
% number - used to balance the columns on the last page
% adjust value as needed - may need to be readjusted if
% the document is modified later
%\IEEEtriggeratref{8}
% The "triggered" command can be changed if desired:
%\IEEEtriggercmd{\enlargethispage{-5in}}

% references section

% can use a bibliography generated by BibTeX as a .bbl file
% BibTeX documentation can be easily obtained at:
% http://mirror.ctan.org/biblio/bibtex/contrib/doc/
% The IEEEtran BibTeX style support page is at:
% http://www.michaelshell.org/tex/ieeetran/bibtex/
%\bibliographystyle{IEEEtran}
% argument is your BibTeX string definitions and bibliography database(s)
%\bibliography{IEEEabrv,../bib/paper}
%
% <OR> manually copy in the resultant .bbl file
% set second argument of \begin to the number of references
% (used to reserve space for the reference number labels box)

\begin{thebibliography}{10}

\bibitem{GraphChi} Aapo Kyrola, Guy Blelloch, and Carlos Guestrin. 2012. GraphChi: large-scale graph computation on just a PC. In Proceedings of the 10th USENIX conference on Operating Systems Design and Implementation (OSDI'12). USENIX Association, Berkeley, CA, USA, 31-46.

\bibitem{Pregel} Malewicz, Matthew H. Austern, Aart J.C Bik, James C. Dehnert, Ilan Horn, Naty Leiser, and Grzegorz Czajkowski. 2010. Pregel: a system for large-scale graph processing. In Proceedings of the 2010 ACM SIGMOD International Conference on Management of data (SIGMOD '10). ACM, New York, NY, USA, 135-146.

\bibitem{PowerGraph} Joseph E. Gonzalez, Yucheng Low, Haijie Gu, Danny Bickson, and Carlos Guestrin. 2012. PowerGraph: distributed graph-parallel computation on natural graphs. In Proceedings of the 10th USENIX conference on Operating Systems Design and Implementation (OSDI'12). USENIX Association, Berkeley, CA, USA, 17-30.

\bibitem{Graphicionado} T. J. Ham, L. Wu, N. Sundaram, N. Satish and M. Martonosi, "Graphicionado: A high-performance and energy-efficient accelerator for graph analytics," 2016 49th Annual IEEE/ACM International Symposium on Microarchitecture (MICRO), Taipei, 2016, pp. 1-13.

\bibitem{Tesseract}J. Ahn, S. Hong, S. Yoo, O. Mutlu and K. Choi, "A scalable processing-in-memory accelerator for parallel graph processing," 2015 ACM/IEEE 42nd Annual International Symposium on Computer Architecture (ISCA), Portland, OR, 2015, pp. 105-117.

\bibitem{GraphR} L. Song, Y. Zhuo, X. Qian, H. Li and Y. Chen, "GraphR: Accelerating Graph Processing Using ReRAM," 2018 IEEE International Symposium on High Performance Computer Architecture (HPCA), Vienna, 2018, pp. 531-543.

\bibitem{GraphP} M. Zhang et al., "GraphP: Reducing Communication for PIM-Based Graph Processing with Efficient Data Partition," 2018 IEEE International Symposium on High Performance Computer Architecture (HPCA), Vienna, 2018, pp. 544-557.

\bibitem{SocialGraph} Muhammad Imran, Carlos Castillo, Fernando Diaz, and Sarah Vieweg. 2015. Processing Social Media Messages in Mass Emergency: A Survey. ACM Comput. Surv. 47, 4, Article 67 (June 2015), 38 pages. 

\bibitem{RoadGraph} C. Unsalan and B. Sirmacek, "Road Network Detection Using Probabilistic and Graph Theoretical Methods," in IEEE Transactions on Geoscience and Remote Sensing, vol. 50, no. 11, pp. 4441-4453, Nov. 2012.

\bibitem{Biscuit} Boncheol Gu, Andre S. Yoon, Duck-Ho Bae, Insoon Jo, Jinyoung Lee, Jonghyun Yoon, Jeong-Uk Kang, Moonsang Kwon, Chanho Yoon, Sangyeun Cho, Jaeheon Jeong, and Duckhyun Chang. 2016. Biscuit: a framework for near-data processing of big data workloads. SIGARCH Comput. Archit. News 44, 3 (June 2016), 153-165.

\bibitem{GraphH} G. Dai et al., "GraphH: A Processing-in-Memory Architecture for Large-scale Graph Processing," in IEEE Transactions on Computer-Aided Design of Integrated Circuits and Systems.

\bibitem{BGA-book} Systems for Big Graph Analytics by Yan, Da, Tian, Yuanyuan, Cheng, James

\bibitem{SG-MT} Gupta, S. (2018). Processing in Memory using Emerging Memory Technologies. UC San Diego. ProQuest ID: 
Merritt ID: ark:/13030/m5bg7kwn. Retrieved from https://escholarship.org/uc/item/95z3z84c

\bibitem{graphFwSurvey} Omar Batarfi, Radwa El Shawi, Ayman G. Fayoumi, Reza Nouri, Seyed-Mehdi-Reza Beheshti, Ahmed Barnawi, and Sherif Sakr. 2015. Large scale graph processing systems: survey and an experimental evaluation. Cluster Computing 18, 3

\bibitem{hmc} J. T. Pawlowski, "Hybrid memory cube (HMC)," 2011 IEEE Hot Chips 23 Symposium (HCS), Stanford, CA, USA, 2011, pp. 1-24. 

\bibitem{amc} R. Nair et al., "Active Memory Cube: A processing-in-memory architecture for exascale systems," in IBM Journal of Research and Development, vol. 59, no. 2/3, pp. 17:1-17:14, March-May 2015.

\bibitem{sram_search} Q. Dong et al., "A 4 + 2T SRAM for Searching and In-Memory Computing With 0.3-V $V_{\mathrm {DDmin}}$," in IEEE Journal of Solid-State Circuits, vol. 53, no. 4, pp. 1006-1015, April 2018.

\bibitem{bulkbit} V. Seshadri et al., "Fast Bulk Bitwise AND and OR in DRAM," in IEEE Computer Architecture Letters, vol. 14, no. 2, pp. 127-131, 1 July-Dec. 2015.

\bibitem{drisa} Shuangchen Li, Dimin Niu, Krishna T. Malladi, Hongzhong Zheng, Bob Brennan, and Yuan Xie. 2017. DRISA: a DRAM-based Reconfigurable In-Situ Accelerator. In Proceedings of the 50th Annual IEEE/ACM International Symposium on Microarchitecture (MICRO-50 '17). ACM, New York, NY, USA, 288-301

\bibitem{imi} T. Finkbeiner, G. Hush, T. Larsen, P. Lea, J. Leidel and T. Manning, "In-Memory Intelligence," in IEEE Micro, vol. 37, no. 4, pp. 30-38, 2017. doi: 10.1109/MM.2017.3211117

\bibitem{prsram} M. Kang, M. Keel, N. R. Shanbhag, S. Eilert and K. Curewitz, "An energy-efficient VLSI architecture for pattern recognition via deep embedding of computation in SRAM," 2014 IEEE International Conference on Acoustics, Speech and Signal Processing (ICASSP), Florence, 2014, pp. 8326-8330.

\bibitem{cam} S. Jeloka, N. B. Akesh, D. Sylvester and D. Blaauw, "A 28 nm Configurable Memory (TCAM/BCAM/SRAM) Using Push-Rule 6T Bit Cell Enabling Logic-in-Memory," in IEEE Journal of Solid-State Circuits, vol. 51, no. 4, pp. 1009-1021, April 2016.

\bibitem{computecache} S. Aga, S. Jeloka, A. Subramaniyan, S. Narayanasamy, D. Blaauw and R. Das, "Compute Caches," 2017 IEEE International Symposium on High Performance Computer Architecture (HPCA), Austin, TX, 2017, pp. 481-492.

\bibitem{sramml} J. Zhang, Z. Wang and N. Verma, "In-Memory Computation of a Machine-Learning Classifier in a Standard 6T SRAM Array," in IEEE Journal of Solid-State Circuits, vol. 52, no. 4, pp. 915-924, April 2017.

\bibitem{6Tsram} Q. Dong et al., "A 4 + 2T SRAM for Searching and In-Memory Computing With 0.3-V $V_{\mathrm {DDmin}}$," in IEEE Journal of Solid-State Circuits, vol. 53, no. 4, pp. 1006-1015, April 2018.

\bibitem{tcamrram} Qing Guo, Xiaochen Guo, Yuxin Bai, and Engin İpek. 2011. A resistive TCAM accelerator for data-intensive computing. In Proceedings of the 44th Annual IEEE/ACM International Symposium on Microarchitecture (MICRO-44). ACM, New York, NY, USA, 339-350.

\bibitem{memres} Dmitri B Strukov, Gregory S Snider, et al. The missing memristor found. Nature, 453(7191):80–83, 2008.

\bibitem{xbarcomp} B. Li, Y. Shan, M. Hu, Y. Wang, Y. Chen and H. Yang, "Memristor-based approximated computation," International Symposium on Low Power Electronics and Design (ISLPED), Beijing, 2013, pp. 242-247.

\bibitem{xbrml} Yongtae Kim, Yong Zhang, and Peng Li. 2015. A Reconfigurable Digital Neuromorphic Processor with Memristive Synaptic Crossbar for Cognitive Computing. J. Emerg. Technol. Comput. Syst. 11, 4, Article 38 (April 2015), 25 pages.

\bibitem{magic} S. Kvatinsky et al., "MAGIC—Memristor-Aided Logic," in IEEE Transactions on Circuits and Systems II: Express Briefs, vol. 61, no. 11, pp. 895-899, Nov. 2014.

\bibitem{imply} S. Kvatinsky, G. Satat, N. Wald, E. G. Friedman, A. Kolodny and U. C. Weiser, "Memristor-Based Material Implication (IMPLY) Logic: Design Principles and Methodologies," in IEEE Transactions on Very Large Scale Integration (VLSI) Systems, vol. 22, no. 10, pp. 2054-2066, Oct. 2014.

\bibitem{magic-add} N. Talati, S. Gupta, P. Mane and S. Kvatinsky, "Logic Design Within Memristive Memories Using Memristor-Aided loGIC (MAGIC)," in IEEE Transactions on Nanotechnology, vol. 15, no. 4, pp. 635-650, July 2016.


\end{thebibliography}

\end{document}


