\section{Graph Processing}

%Bring out the importance of Graph processing from \cite{BrainGraph,RoadGraph, SocialGraph} and discuss state-of-the-art systems. Take the example of SSSP application and discuss how the Implementation changes with various models.

Current graph analytics frameworks are based on two fundamentally different computation models - Vertex-centric and Edge-centric processing. Among these two, Vertex-centric programming models are prominent as they are more intuitive than the other.   Matrix Algebra-based systems which performs operations on the adjacency matrix of a graph is another intuitive model for graph processing.

\subsection{Vertex-Centrix Processing}
In Vertex-Centrix processing frameworks, the vertices are treated as first class objects and edges as association between two vertices. Vertex-Centric progrmming model provides an intuitive framework to construct a large scale graph program based on the computation on a single vertex. Two primary categories of the vertex-centric programming model are summarised below.

\subsubsection{Pregel}
Pregel\cite{Pregel} is a vertex-centric programming model following the Bulk Synchronous Parallel style of execution. The computation on vertices are represented as a sequence of supersteps with a global synchronisation among the vertices participating in a superstep. A superstep only contains the set of active nodes which was enabled by the vertices of previous superstep. The vertices in a superstep can be operated on in parallel and they generate messages to vertices of the next superstep. The program halts when there are no active vertices in the next superstep. Algorithm \ref{vertex_algo} is a pseudo code of a typical Pregel program. The Synchronicity of the model makes it easy to reason about and avoids any possibility deadlock or data races. One limitation with this model is the presence of the global barrier synchronisation, whose performance effects can be mitigated by having adequate parallelism slack. Some of the frameworks that are based on the Pregel programming are Giraph, Giraph++, Mizan, GPS, Pregelix and Pregel+.

\begin{algorithm}
\caption{Pregel Abstraction}
\label{vertex_algo}
\begin{algorithmic}[1]
  \State super\_step $\gets 0$
  \While {active\_vertices $\ne \emptyset$}
    \For{v $\in$ active\_vertices}
      \State v.value $\gets$ v.compute(msgs) 
      \State v.send\_message(neighbourhood(v))
      \State v.halt()
    \EndFor
  \EndWhile
  \State super\_step $\gets$ super\_step + $1$

\end{algorithmic}
\end{algorithm}

\subsubsection{Gather-Apply-Scatter}
GAS \cite{PowerGraph} programming model splits graph processing into three steps - Gather, Apply and Scatter. During the Gather phase a vertex accumulates information about its adjacent edges and processes it. The processed value is used to update itself and its neighbors in Apply and Scatter phase respectively. The updated neighbors are activated and will go through the GAS steps. The program halts when there are no updated vertices. The GAS model can be realised as both synchronous and asynchronous execution depending on the requirements. One fundamental difference from pregel is that vertices can directly access the neighborhood without having to be explicitly communicated to it. As a result, this requires shared memory abstraction for the GAS model to be realised, which increases the memory access latency in a distributed systems. Techniques such as data replication, prefetching, etc. can be employed to overcome the bottleneck. A basic sequence of steps involved in GAS model is presented in Algorithm \ref{gas_algo}. GraphLab, GraphChi and PowerGraph are some of the frameworks developed on GAS Abstraction. 

The paper \cite{graphFwSurvey} does a performance evaluation of the various GAS and Pregel based large scale graph processing framework. GAS models enables data partitioning based on edges whereas pregel model is limited to vertex based partitioning. As many natural graphs follow power law like vertex-degree distribution, edge-based graph partitioning leads to better load balancing for such graphs.

\begin{algorithm}
\caption{GAS Abstraction}
\label{gas_algo}
\begin{algorithmic}[1]
  \State super\_step $\gets 0$
  \While {active\_vertices $\ne \emptyset$}
    \For{v $\in$ active\_vertices}
%      \State \texttt{\small /* Pull data from replicas */}
      \State accumulator $\gets$ sum(v.gather(neighbourhood(v))
%      \State \texttt{\small /* Push updates to replicas */}
      \State v.value $\gets$ v.apply(accumulator, val(v))
      \State v.scatter(neighbourhood(v))
    \EndFor
  \EndWhile
  \State super\_step $\gets$ super\_step + $1$

\end{algorithmic}
\end{algorithm}

\subsection{Matrix Algebra-Based Systems} \label{gmat}
Before the advent of graph style processing frameworks such as Pregel, graph computations were modelled as matrix-vector multiplication. Every iteration executes the user defined function on the adjacency list(vector) of a vertex and update the adjacency matrix. Such matrix based computation enables the use of optimisations such as efficient data layouting, prefetching and use of standard linear algebra sub-routines. PEGASUS, GBASE and SystemML are three frameworks which incorporate matrix based graph computation in their framework. One drawback of matrix based graph computation is the amount of redundant work that is performed every step. Each iteration triggers a full matrix computation even though only a few of the node update their value. In contrast, in pregel like systems, computation is performed only on the active nodes which are bound to update their values. To overcome the limitation, some frameworks present a vertex-centrix programming framework and translate it to matrix computation underneath to exploit the best of both the worlds.