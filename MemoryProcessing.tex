\section{In/Near-Memory Processing}
- Reduce the cost of data movement
- PIM Analog chracteristics


  \subsection{Near-Data Processing}
  Discuss \cite{Biscuit} like architectures where the computations are moved closer to the memory elements but not inside them.
  End the section by indicating that more bandwidth can be exploited when the processing is integrated with memory.
  - User managed memory layouting graphicianado, biscuit, etc
  
  \subsection{In-memory Processing}
    Non-vonneuman architectures
    \subsubsection{SRAM-PIM}
    
    \subsubsection{DRAM-PIM}
      Present a basic idea of the \cite{AMC} paper which proposes a architecture for processing-in-memory architecture built on commercially demonstrated Hybrid Memory Cube. Also highlight the loopholes/assumptions with such design. Also list other architectures, like GraphH\cite{GraphH}, which builds on the technology.

    \subsubsection{NVM-PIM}
      Briefly list the problems with DRAM in terms of energy and performance, and bring out NVMs as an alternative to DRAM.
      \cite{MagicNOR} introduces a efficient NOR operation which can be performed on the bitlines of ReRAM based memory. Show example architecture of NVM PIN and explain it working with relevant figures.
