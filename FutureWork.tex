\section{Completed and Future Work}

The architectures listed above were proposed in the last three years, with the advent of new memory technologies which made in-memory processing feasible. An in-depth study of the hardware and software stack of these PIM-based accelerators and their design choices, may expose untapped design points to improve performance. The plan was to begin by understanding various design points listed in Section \ref{DesignSpace}, evaluate the accelerators for strengths and limitations, and identify potential improvements in their design, leading to a survey paper or an original research paper.

I have read papers GraphR\cite{GraphR}, GraphP\cite{GraphP} and other papers based on NVM-PIM \cite{k-means}. Apart from that I also ramped up with Pregel and ZSim Simulation framework. The next step would be to prototype one of the architectures (possibly GraphR or GraphH) and identify bottlenecks to improve the perforamance. A survey paper on this topic makes little sense as the in-memory processing of graphs is a relatively narrow and emerging field. Moreover, there are recent resources \cite{BGA-book, SG-MT} that contain an exhaustive analysis of the topic.
