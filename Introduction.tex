\begin{abstract}
 
  Graph algorithms are ubiquitous, from social science to machine learning analysis. They are also notorious for performing poorly in state-of-the-art computing systems, requiring more compute capability than they need. This is due to the memory bandwidth limitations of the current computing systems. In this regard, the Processing-In-Memory technology shows potential in tackling the bottleneck. A few graph analytics accelerators have been proposed based on the in-memory computing technology in recent years. This paper attempts to summarize the basic graph programming models, various in-memory technologies, and finally discuss the recently developed graph accelerator's endeavor at alleviating various performance bottlenecks.
  
\end{abstract}


\section{Introduction}
% no \IEEEPARstart
Graph algorithms are an interesting class of application for which a single mac mini desktop equipped with SSD outperforms a medium sized cluster! \cite{GraphChi} The primary reason for such a counter-intuitive behavior can be attributed to the irregular data-access pattern inherent in those algorithms. They also have a low computation to communication ratio which is exacerbated in the modern memory-bandwidth limited systems. However, graph algorithms are highly iterative and there is abundance of parallelism that can be exploited. These idiosyncrasies of graph algorithm render commercial HPC systems inefficient or unsuitable for accelerating such applications. In other words, there's a lot of scope for deriving performance improvements from a processor customised for graph processing. Extensive research on the topic in the past decade have revealed new insights on the algorithmic, programming model and hardware aspects of such systems. From the hardware perspective, emergence of in-memory computing technology has enabled us to overcome the limitation of the conventional Von-Neumann architecture. Logic units can now be integrated into the memory cells, evading the dreaded off-chip communication. Such technologies have given birth to Processing-In-Memory (PIM) based accelerators for graph analytics. 

The rest of the paper is organised as follows. Two basic graph processing techniques are discussed in Section 2. Section 3 briefly introduces PIM techniques on different memory technologies. Section 4 discusses some of the state-of-the-art graph analytics accelerators building on the technologies described in Section 2 \& 3.
